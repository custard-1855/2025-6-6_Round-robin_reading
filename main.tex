\documentclass[unicode,12pt,aspectratio=169,dvipdfmx]{beamer}
\usepackage{bxdpx-beamer}
\usetheme[progressbar=frametitle]{metropolis}
\renewcommand{\kanjifamilydefault}{\gtdefault}
\usepackage{bm}

\title{介護における音響HARと連合学習を用いた異常検知}
% \title{「A Survey on Federated Learning in Human Sensing」について}
\author{竹本志恩}
\institute{INIAD}
\date{\today}
\subject{研究報告 / 輪読}
\AtBeginSection[]{
  \begin{frame}[plain]
    \frametitle{目次}
    \tableofcontents[currentsection, hideallsubsections]
  \end{frame}
}

% \usepackage[utf8]{inputenc}
% \usepackage{pxlatex} % 日本語表示用
% \usepackage[whole]{kumacontents} % 引用番号の表示用
% \usepackage{amsmath} % 数式用

% --- セクション開始時に目次を表示 ---
% 発表の区切りで目次を挿入し、聴衆が構成を把握しやすくします ([5])。
\AtBeginSection[]{%
  \frame{\frametitle{目次}\tableofcontents[currentsection, hideallsubsections]}%
}

% --- 最終スライドの定義 (オプション) ---
% \appendix 環境や専用フレームで、発表終了を示すスライドを追加できます ([5])。
% 例:
% \newcommand{\thanksframe}{
%   \begin{frame}[plain] % plainオプションでヘッダー/フッターを非表示
%     \centering
%     \vfill % 垂直方向中央揃え
%     \includegraphics[height=2cm]{logo.png}\par % ロゴなどがあれば
%     \vspace{1em}
%     {\Huge ご清聴ありがとうございました}\par
%     \vspace{1em}
%     {\Large 質問等ございましたらお気軽にどうぞ.}\par
%     \vspace{2em}
%     {\large 連絡先: your.email@example.com}\par
%     \vfill
%   \end{frame}
% }


\begin{document}

% タイトルスライド
\begin{frame}
    \titlepage
\end{frame}

% 各セクション
\section{はじめに}
\begin{frame}{はじめに}
    \begin{itemize}
        \item 発表論文の概要
        \begin{itemize}
            \item タイトル: A Survey on Federated Learning in Human Sensing \cite{Source52, Source53}
            \item 著者: Mohan Li, Martin Gjoreski, Pietro Barbiero, Gašper Slapničar, Mitja Luštrek, Nicholas D. Lane, Marc Langheinrich \cite{Source52}
            \item 出典: ACM, 2025年1月公開 \cite{Source53, Source54}
            \item 内容: \textbf{Human Sensing分野におけるFederated Learning (FL) の応用}に関する包括的サーベイ \cite{Source52, Source53}。現状、課題、分類、今後の研究方向を提示 \cite{Source52, Source53}。
        \end{itemize}
    \end{itemize}
\end{frame}


\begin{frame}{はじめに}
    \begin{itemize}
        \item Human SensingとMLにおけるプライバシー課題
        \begin{itemize}
            \item Human Sensingは、センサー技術の発展とウェアラブルデバイスの普及により進化 \cite{Source54}。
            \item 人間の活動、生理心理状態、環境との相互作用を監視し、生活の質向上に貢献 \cite{Source52, Source54}。
            \item しかし、その基盤となる詳細かつ\textbf{プライバシーに敏感なデータ}は、法的な規制や倫理的な懸念を引き起こす \cite{Source52, Source55}。
        \end{itemize}
        \item FLがこの課題をどう解決するか
        \begin{itemize}
            \item FLは、\textbf{生データを中央サーバーに送らず}に正確なMLモデルを構築できる \cite{Source55}。
            \item これにより、多くのプライバシー懸念を緩和する可能性を持つ \cite{Source52, Source55}。
        \end{itemize}
    \end{itemize}
\end{frame}



\section{本サーベイの貢献と主要ドメイン}
\begin{frame}{本サーベイの貢献}
    \begin{itemize}
        \item Human SensingにおけるFLに特化した包括的サーベイ
        \begin{itemize}
            \item 既存のFLサーベイはIoTや医療(IoMT)、推薦システムなどに焦点を当てている \cite{Source62, Source64}。
            \item \textbf{本サーベイはHuman Sensingに特化}している点が特徴 \cite{Source64}。
        \end{itemize}
        \item FLの応用評価のための\textbf{8次元フレームワーク}を提案 \cite{Source56, Source57}
        \begin{itemize}
            \item プライバシーとセキュリティ、通信コスト、システム異質性、統計的異質性、ラベルなしデータ使用、Simplified Setup(簡略化された望ましくない設定)、サーバー最適化FL、クライアント最適化FL \cite{Source56, Source57}。
        \end{itemize}
        \item 応用指向の\textbf{分類(6つのドメイン)}を提示 \cite{Source57, Source61}
        \begin{itemize}
            \item Audio and Speech Processing, Well-being, User Identification, Human Mobility and Localization, Activity Recognition (HAR), Interface Development \cite{Source57, Source61}。
        \end{itemize}
    \end{itemize}
\end{frame}

\begin{frame}{主要な応用ドメイン}
    \begin{itemize}
        \item 本サーベイが特定した6つの主要な応用ドメイン \cite{Source57, Source61}
        \item \textbf{Activity Recognition (HAR) が最も研究が多い分野}
        \begin{itemize}
            \item サーベイ対象論文の\textbf{31.6\%}を占める \cite{Source101}。
            \item Well-being (21.4\%)、User Identification (15.3\%)、Human Mobility and Localization (14.9%)、Audio and Speech Processing (13.5\%) が続く \cite{Source101}。
            \item Interface Development は最も少ない (3.3\%) \cite{Source100, Source101}。
        \end{itemize}
        \item このことから、\textbf{HAR分野はFLの主要な応用先として注目されている}ことがわかる \cite{Source101}。
    \end{itemize}
\end{frame}

\section{Human Activity Recognition (HAR) とは}
\begin{frame}{Human Activity Recognition (HAR) とは}
    \begin{itemize}
        \item HARの定義と重要性
        \begin{itemize}
            \item センサー読み取り値から\textbf{人間の活動タイプを認識}する問題 \cite{Source222}。
            \item ヘルスケア、スマートホーム、リハビリテーション、転倒検知など、幅広い分野で重要な役割を果たす \cite{Source2, Source6}。
        \end{itemize}
        \item 主要なセンサータイプ
        \begin{itemize}
            \item \textbf{ウェアラブルセンサー}: スマートウォッチ、スマートフォン、リストバンドなど (加速度計、ジャイロスコープなど) \cite{Source9}。小型、低コスト、柔軟な装着が可能 \cite{Source445}。
            \item \textbf{環境センサー}: 環境に設置されたセンサー (ドアスイッチ、圧力センサー、マイクなど) \cite{Source13}。ユーザーにとって負担が少ない \cite{Source14}。
            \item \textbf{カメラベースシステム}: ビデオや画像を使用 \cite{Source219}。直接的で正確な情報を提供できる \cite{Source219}。
        \end{itemize}
        \item 従来のHARにおける課題
        \begin{itemize}
            \item \textbf{プライバシー懸念}: 特にカメラベースシステムや中央集権的なデータ収集 \cite{Source436, Source440}。
            \item \textbf{通信コスト}: 大量のセンサーデータを中央サーバーに送信する際の負担 \cite{Source440}。
            \item \textbf{手作業による特徴エンジニアリング}: 従来のML手法ではドメイン知識に依存し、時間と労力がかかる \cite{Source221}。
            \item \textbf{限られたデータセット}: プライバシー懸念から大規模な実世界データセットが少ない \cite{Source419}。
        \end{itemize}
        \item 深層学習の貢献
        \begin{itemize}
            \item 手作業による特徴抽出が不要になり、\textbf{高レベルで意味のある特徴を自動学習}できる \cite{Source221, Source381}。
            \item 精度とロバスト性が向上 \cite{Source221, Source381}。
        \end{itemize}
    \end{itemize}
\end{frame}

\section{HARにおけるFLの導入と想定シナリオ}
\begin{frame}{HARにおけるFLの導入}
    \begin{itemize}
        \item FLによる課題解決の方向性 \cite{Source410, Source416}
        \begin{itemize}
            \item \textbf{プライバシー保護}: \textbf{ユーザーの生データはローカルデバイスに保持}し、モデルの更新情報のみを共有する \cite{Source410, Source416}。
            \item \textbf{通信コスト削減}: 生データと比較してモデル更新情報のサイズが小さい場合 \cite{Source254, Source410}。
            \item \textbf{分散型学習}: 大量のデータを収集・転送・保存する必要がない \cite{Source410}。
        \end{itemize}
        \item HARにおけるFL研究の活発化
        \begin{itemize}
            \item サーベイ論文でも多くの研究がHAR分野に集中している \cite{Source101}。
            \item \textbf{現実世界の複雑な課題}(データ異質性、システム異質性など)への対応が検討されている \cite{Source55}。
        \end{itemize}
    \end{itemize}
\end{frame}

\begin{frame}{HAR-FL想定シナリオ:介護施設での見守りシステム}
    \begin{itemize}
        \item 目的:\textbf{高齢者の異常行動を早期に検知し、介護者に通知} \cite{Source515}。
        \begin{itemize}
            \item 特に、\textbf{転倒、異様な咳き込み、苦痛の声}など、緊急性の高いイベントの検知 \cite{Source516, Source519}。
        \end{itemize}
        \item 背景:高齢化社会における介護人材不足の深刻化 \cite{Source2}。安価で効果的な見守りシステムの需要 \cite{Source523}。
        \item 使用技術
        \begin{itemize}
            \item \textbf{音によるHAR}: 各部屋のIoTデバイスで環境音を収集し、音響イベントを検出 \cite{Source516}。
            \item \textbf{行動認識}: 音響イベントやその時系列から利用者の行動(転倒、呼吸困難など)を判断 \cite{Source516}。
            \item \textbf{Federated Learning (FL)}: 各部屋のIoTデバイスをクライアントとし、モデルを学習・更新 \cite{Source516}。
        \end{itemize}
        \item FL導入の利点 \cite{Source516}
        \begin{itemize}
            \item \textbf{プライバシー保護}: \textbf{利用者の環境音(生データ)は部屋から出さない} \cite{Source516}。プライベートな会話などの漏洩リスクを低減。
            \item \textbf{パーソナル化}: 各部屋の環境や利用者の生活パターンに合わせたモデルを学習し、精度向上 \cite{Source418, Source514}。
            \item \textbf{スケーラビリティ}: クライアント(部屋)が増えても対応しやすい分散学習 \cite{Source517}。
        \end{itemize}
    \end{itemize}
\end{frame}

\begin{frame}{想定される課題と検討事項(シナリオより)}
    \begin{itemize}
        \item \textbf{緊急性の高いイベントデータの収集とアノテーション}:
        \begin{itemize}
            \item これらのデータは稀であるため、収集・準備が課題 \cite{Source50, Source521}。
            \item オープンデータ(SAFE, DESED, AudioSetなど)の活用や合成データ生成などの検討 \cite{Source520, Source547}。
        \end{itemize}
        \item \textbf{データ異質性 (Non-IID)}:
        \begin{itemize}
            \item 部屋ごと、利用者ごとに発生する音響イベントの種類や頻度、音響特性が異なる \cite{Source521}。
            \item クライアントごとのデータ分布のばらつきに対応が必要 \cite{Source521}。
        \end{itemize}
        \item \textbf{クライアントの異質性}:
        \begin{itemize}
            \item IoTデバイスの計算能力、メモリ、接続性などが異なる \cite{Source55, Source521}。
            \item 軽量なモデルや連合分割学習(FSL)などが有効 \cite{Source521}。
        \end{itemize}
        \item \textbf{リアルタイム性能}:
        \begin{itemize}
            \item 異常検知には即時性が求められる \cite{Source308}。FLの学習・推論プロセス全体のレイテンシ検討。
        \end{itemize}
        \item \textbf{モデルの汎化}:
        \begin{itemize}
            \item 特定の環境で学習したモデルが、別の部屋や新規利用者に適用できるか \cite{Source418}。
            \item メタ学習や表現学習を用いたアプローチが有望 \cite{Source418}。
        \end{itemize}
    \end{itemize}
\end{frame}

\section{HAR-FLの主要課題とFLの対応(サーベイより)}
\begin{frame}{HAR-FLにおける主要な課題とFLの対応}
    \begin{itemize}
        \item \textbf{データ異質性 (Non-IID)} \cite{Source544, Source55}
        \begin{itemize}
            \item 課題: クライアント間で活動タイプやセンサーデータ(信号分布)に大きな偏りがある \cite{Source55, Source418, Source521}。従来のFedAvgでは性能が劣化しやすい \cite{Source55, Source420}。
            \item FLによる対応例: \textbf{パーソナル化FL (pFL)} \cite{Source544} (Meta-HAR \cite{Source418}、FedDL \cite{Source254})、クラスタリング \cite{Source143}、知識蒸留 \cite{Source90}。
        \end{itemize}
        \item \textbf{限られたラベルデータ/ラベルなしデータ} \cite{Source55, Source544}
        \begin{itemize}
            \item 課題: ラベル付けのコストや希少なイベントのため、十分なラベル付きデータがない \cite{Source55, Source521}。
            \item FLによる対応例: \textbf{半教師あり学習} \cite{Source15, Source96}、教師なし学習 \cite{Source96}、自己教師あり学習 \cite{Source96}。
        \end{itemize}
        \item \textbf{通信コスト} \cite{Source55, Source544}
        \begin{itemize}
            \item 課題: エッジデバイスの帯域幅やバッテリーの制約 \cite{Source55, Source521}。
            \item FLによる対応例: \textbf{動的レイヤー共有 (FedDL)} \cite{Source254, Source256}、モデル圧縮・量子化 \cite{Source254}。
        \end{itemize}
    \end{itemize}
\end{frame}

\begin{frame}{HAR-FLにおける主要な課題とFLの対応 (続き)}
    \begin{itemize}
        \item \textbf{プライバシー/セキュリティ} \cite{Source55, Source544}
        \begin{itemize}
            \item 課題: FLでも勾配からの情報漏洩やポイズニング攻撃などのリスク \cite{Source103, Source522}。
            \item FLによる対応例: \textbf{差分プライバシー (DP)} \cite{Source303, Source348}、\textbf{セキュアアグリゲーション} \cite{Source344, Source544}、クライアントフィルタリング \cite{Source107}、連合分割学習(FSL) \cite{Source348}。
        \end{itemize}
        \item \textbf{システム異質性} \cite{Source55, Source544}
        \begin{itemize}
            \item 課題: クライアントデバイスの計算能力、メモリ、接続性などが異なる \cite{Source55, Source521}。
            \item FLによる対応例: リソースアウェアなクライアント選択や学習手法 \cite{Source109, Source110}。
        \end{itemize}
    \end{itemize}
\end{frame}

\section{まとめ}
\begin{frame}{まとめ}
    \begin{itemize}
        \item Human Sensingは、MLの活用により大きく進展する一方、\textbf{データプライバシーという重要な課題}を抱えている \cite{Source52, Source55}。
        \item Federated Learningは、データをローカルに保持したまま学習を進めることで、このプライバシー課題に対する\textbf{有望な解決策}を提供する \cite{Source52, Source55}。
        \item Human Activity Recognition (HAR) 分野は、FLの応用が最も活発に行われている分野の一つ \cite{Source101}。
        \item \textbf{介護施設での見守りシステム}のような具体的なシナリオにおいて、HAR-FLはプライバシーを保護しつつ、\textbf{利用者一人ひとりに合わせた高精度な異常検知}を実現する可能性を秘めている \cite{Source515, Source543}。
        \item 今後の研究は、より現実的な条件下での性能評価、多様なモダリティの統合、そしてプライバシー保護技術のさらなる発展に焦点を当てることで、HAR-FLの\textbf{社会実装を加速}させることが期待される \cite{Source105}。
    \end{itemize}
\end{frame}

\end{document}
