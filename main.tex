\documentclass[unicode,12pt,aspectratio=169,dvipdfmx]{beamer}
\usepackage{bxdpx-beamer}
\usetheme[progressbar=frametitle]{metropolis}
\renewcommand{\kanjifamilydefault}{\gtdefault}
\usepackage{bm}
\title{介護における音響HARと連合学習を用いた異常検知}
\author{発表者名(竹本志恩様の例に倣い、発表者名はプレースホルダーとしました)}
\date{\today}
\institute{所属機関名(INIAD様の例に倣い、所属機関名はプレースホルダーとしました)}
\subject{研究報告 / 輪読}
\AtBeginSection[]{
  \begin{frame}[plain]
    \frametitle{目次}
    \tableofcontents[currentsection, hideallsubsections]
  \end{frame}
}
\begin{document}
%%%%%%%%%%%%%%%%%%%%%%
% 1. タイトルページ
%%%%%%%%%%%%%%%%%%%%%%
\frame{\maketitle}
%%%%%%%%%%%%%%%%%%%%%%
% 2. 目次
%%%%%%%%%%%%%%%%%%%%%%
\begin{frame}{目次}
  \tableofcontents[hideallsubsections]
\end{frame}
%%%%%%%%%%%%%%%%%%%%%%%%%%%%%%%%%%%%%%%%%%%%%%%%
% 3. はじめに
%%%%%%%%%%%%%%%%%%%%%%%%%%%%%%%%%%%%%%%%%%%%%%%%
\section{はじめに:背景と目的}
\begin{frame}{背景と目的}
  \begin{itemize}
    \item \textbf{高齢化社会の課題}
    \begin{itemize}
      \item 遠隔・継続的な見守りの重要性
      \item 単身高齢者の安全/迅速な介入のニーズ
    \end{itemize}
    \item \textbf{従来技術の限界}
    \begin{itemize}
      \item カメラ等はプライバシー懸念
      \item 音響による非侵襲的なモニタリングに期待
    \end{itemize}
    \item \textbf{技術的着眼点}
    \begin{itemize}
      \item \textbf{音響HAR}:環境音から活動・異常検知
      \item \textbf{連合学習(FL)}:分散データ、プライバシー保護学習
    \end{itemize}
    \item \textbf{本発表の目的}
    \begin{itemize}
      \item 音響HAR×FLによる介護分野の異常検知可能性を整理・考察
    \end{itemize}
  \end{itemize}
\end{frame}
%%%%%%%%%%%%%%%%%%%%%%%%%%%%%%%%%%%%%%%%%%%%%%%%%%%%%%%%%%
% 4. 関連研究・背景
%%%%%%%%%%%%%%%%%%%%%%%%%%%%%%%%%%%%%%%%%%%%%%%%%%%%%%%%%%
\section{関連研究・背景}
\begin{frame}{関連研究・背景}
  \begin{itemize}
    \item \textbf{人間行動認識(HAR)}
    \begin{itemize}
      \item センサー/映像/音響で人間の活動を推定
      \item 手動特徴抽出や浅層学習の限界
    \end{itemize}
    \item \textbf{連合学習(FL)}
    \begin{itemize}
      \item 分散データ×グローバルモデル(FedAvg等)
      \item データ集中/プライバシー/非IID問題に対応
    \end{itemize}
    \item \textbf{HAR×FLの関連研究/サーベイ}
    \begin{itemize}
      \item 深層HAR、FL in Human Sensing 等、多数のアップデートが進行中
    \end{itemize}
  \end{itemize}
\end{frame}
%%%%%%%%%%%%%%%%%%%%%%%%%%%%%%%%%%%%%%%%%%%%%%%%%%%%%%%%%%
% 5. 技術要素と手法
%%%%%%%%%%%%%%%%%%%%%%%%%%%%%%%%%%%%%%%%%%%%%%%%%%%%%%%%%%
\section{技術要素と手法}
\begin{frame}{重要な音響イベント \& 特徴抽出}
  \begin{columns}[T]
    \column{0.55\textwidth}
      \textbf{介護に重要な音響イベント例:}
      \begin{itemize}
        \item 転倒音、悲鳴・うめき声、助けを求める声
        \item 呼吸音、異常な咳、ガラス破損音
        \item 「いつもと違う」生活音パターン
      \end{itemize}
    \column{0.45\textwidth}
      \textbf{音響特徴抽出:}
      \begin{itemize}
        \item MFCC:定番、オンデバイスにも適用例多
        \item メルスペクトログラム:CNN等に適
        \item 特徴選択自体もプライバシー影響
      \end{itemize}
  \end{columns}
\end{frame}
\begin{frame}{連合学習アルゴリズム・プライバシー技術}
  \begin{itemize}
    \item \textbf{FLアルゴリズム}
      \begin{itemize}
        \item FedAvg: ベースライン
        \item パーソナライズFL: 個別適応(Meta-HARなど)
        \item 連合分割学習: サーバ/クライアント分割
      \end{itemize}
    \item \textbf{プライバシー強化技術(PETs)}
      \begin{itemize}
        \item 差分プライバシー(DP)、セキュアアグリゲーション(SA)、準同型暗号(HE)
        \item モデル有用性⇔計算/通信コストのトレードオフ
      \end{itemize}
  \end{itemize}
\end{frame}
%%%%%%%%%%%%%%%%%%%%%%%%%%%%%%%%%%%%%%%%%%%%%%%%%%%%%%%%%%
% 6. 課題
%%%%%%%%%%%%%%%%%%%%%%%%%%%%%%%%%%%%%%%%%%%%%%%%%%%%%%%%%%
\section{課題}
\begin{frame}{現状の課題}
  \begin{itemize}
    \item \textbf{データ課題}
      \begin{itemize}
        \item 緊急事態データの希少性
        \item アノテ精度/データ不均衡/実データ不足
        \item 非IIDなデータ/ドメインシフト
      \end{itemize}
    \item \textbf{技術課題}
      \begin{itemize}
        \item ノイズ耐性・リアルタイム性
        \item エッジ機器での計算・通信リソース制約
      \end{itemize}
    \item \textbf{プライバシー・倫理課題}
      \begin{itemize}
        \item 勾配漏洩対策/PETs導入コスト
        \item ユーザー説明性、公平性、同意取得
      \end{itemize}
  \end{itemize}
\end{frame}
%%%%%%%%%%%%%%%%%%%%%%%%%%%%%%%%%%%%%%%%%%%%%%%%%%%%%%%%%%
% 7. 将来展望
%%%%%%%%%%%%%%%%%%%%%%%%%%%%%%%%%%%%%%%%%%%%%%%%%%%%%%%%%%
\section{将来展望}
\begin{frame}{将来展望}




22:19
\begin{itemize}
    \item \textbf{データ拡充}
      \begin{itemize}
        \item 実データ+合成データの活用
        \item 文脈情報まで加味したアノテ・生成技術
      \end{itemize}
    \item \textbf{FL/PETs・技術進化}
      \begin{itemize}
        \item 音響に特化した効率的PETs
        \item 非IIDへの頑健なFL手法
        \item エッジ最適化
      \end{itemize}
    \item \textbf{XAI/説明性・ユーザビリティ}
      \begin{itemize}
        \item 音響SEDモデル向けのXAI統合
        \item ユーザー視点での説明/設計
      \end{itemize}
    \item \textbf{マルチモーダル融合/倫理設計}
      \begin{itemize}
        \item 環境/人感センサなど多要素統合
        \item プライバシー・バイ・デザイン等、国際基準も参照
      \end{itemize}
  \end{itemize}
\end{frame}
%%%%%%%%%%%%%%%%%%%%%%%%%%%%%%%%%%%%%%%%%%%%%%%%%%%%%%%%%%
% 8. 結論
%%%%%%%%%%%%%%%%%%%%%%%%%%%%%%%%%%%%%%%%%%%%%%%%%%%%%%%%%%
\section{結論}
\begin{frame}{結論}
  \begin{itemize}
    \item 音響HARとFLは、介護におけるプライバシー保護×異常検知へ大きな可能性
    \item 転倒・痛み・呼吸音など重要音響の的確な識別が鍵
    \item 成功には技術・データ・倫理の\textbf{三位一体の最適化}が必要
    \item 今後、\textbf{データ拡充/FL最適化/XAI/多モーダル/倫理}軸で研究深化を
    \item これらを通じ、「現場で使える見守り」へ進化可能と期待
  \end{itemize}
\end{frame}
%%%%%%%%%%%%%%%%%%%%%%%%%%%%%%%%%%%%%%%%%%%%%%%%%%%%%%%%%%
% 9. 参考文献(必要時)
%%%%%%%%%%%%%%%%%%%%%%%%%%%%%%%%%%%%%%%%%%%%%%%%%%%%%%%%%%
% \begin{frame}{参考文献}
%   \nocite{*}
%   \bibliographystyle{plain}
%   \bibliography{references}
% \end{frame}
\end{document}